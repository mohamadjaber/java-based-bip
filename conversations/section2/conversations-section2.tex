% *********************************************************************
% *********************************************************************

\section{Externalised Approach}
\label{sec:external}

% *********************************************************************

\subsection{Request-Grant Labelled Transition Systems}
\label{sec:external:model}

\begin{definition}[RG-LTS and interactions]
\label{defn:lts}
A \emph{Request-Grant Labelled Transition System} (RG-LTS) is a quadruple
$(Q,\pin,\pout,\goesto[])$, where $Q$ is a finite set of states, $\pin \cap
\pout = \emptyset$ are finite sets of respectively {\em grant} and {\em
  request} ports and $\goesto[] \subseteq Q \times 2^P \times Q$ (with $P =
\pin \cup \pout$) is a transition relation.  Transitions are labelled by
\emph{interactions} $\emptyset \neq a \subseteq P$.  We define $q
\goesto[a] q' \bydef{=}\Big((q, a, q') \in \goesto[]\Big)$ and $q
\goesto[a]\, \bydef{=}\Big(\exists q' \in Q: q \goesto[a] q'\Big)$.  
\end{definition}

\begin{definition}[Behaviour and conversations]
\label{defn:components}
An RG-LTS $(Q, \pin, \pout, \goesto)$ is 
\begin{itemize}
\item a {\em behaviour} if $\pin  = \emptyset$;
\item a \emph{conversation} if, for any transition $q \goesto[a] q'$, both
  $a \cap \pin \neq \emptyset$ and $a \cap \pout \neq \emptyset$.
\end{itemize}
\end{definition}

Conversations represent sequences of interactions. They are used to compose
behaviours.

To simplify the presentation, below we postulate that for any $q$, the
predicate $q \goesto[\emptyset] q$ is true, whenever it appears in the
premises of an SOS rule.

\begin{definition}
  \label{defn:independent}
  A set of RG-LTS $T_i = (Q_i, \pin_i, \pout_i, \goesto)$, for $i \in
  [1,n]$, is called {\em independent} iff their ports are pairwise
  disjoint, \ie[,] denoting $P_i \bydef{=} \pin_i \cup \pout_i$, we have
  $\forall i \neq j,\ P_i \cap P_j = \emptyset$.
\end{definition}

\begin{definition}[1-composable RG-LTS]
  \label{defn:composable}
  Let $T_i = (Q_i, \pin_i, \pout_i, \goesto)$, for $i \in [1,n]$, be a set
  of RG-LTS.  The set $\setdef{T_i}{i\in[1,n]}$ is {\em 1-composable} if
  there exists a partition $\{G, R\}$ of $[1,n]$ (that is $G \cup R =
  [1,n]$ and $G \cap R = \emptyset$), such that both sets
  $\setdef{T_i}{i\in G}$ and $\setdef{T_i}{i\in R}$ are independent and the
  following two conditions hold:
  \begin{enumerate}
  \item $\displaystyle\bigcup_{i \in G} \pout_i\ \cap\ \bigcup_{j \in R}
    P_j = \emptyset$,\nb{$P_i$, not $\pout_i$} that is none of the request
    ports of the RG-LTS in group $G$ belong to any of the RG-LTS in group
    $R$; and
  \item $\displaystyle\bigcup_{i \in G} \pin_i\ \cap\ \bigcup_{j \in R}
    \pin_j = \emptyset$, that is the grant ports of RG-LTS in group $G$
    that also belong to RG-LTS in group $R$ must necessarily be request
    ports in the latter.
  \end{enumerate}
\end{definition}

\begin{definition}[Composition with one conversation]
  \label{defn:composition}
  Let $T_i = (Q_i, \pin_i, \pout_i, \goesto)$, for $i \in [1,n]$, be a set
  of RG-LTS with pairwise disjoint sets of ports, and let $C = (Q_0,
  \pin_0, \pout_0, \goesto)$ be a conversation, such that the set of RG-LTS
  $\{C\}\cup \setdef{T_i}{i\in [1,n]}$ is 1-composable.  As above, we
  denote $P_i \bydef{=} \pin_i \cup \pout_i$, for $i \in [0,n]$.  The
  composition of $T_i$ with $C$ is an RG-LTS given by $C(T_1,\dots,T_n)
  \bydef{=} (Q, \pin, \pout, \goesto)$, where
  \begin{equation}
    \label{eq:composition:ports}
    Q = \prod_{i=0}^n Q_i\,,
    \hspace{1cm}
    \pin = \bigcup_{i=0}^n \pin_i \setminus \bigcup_{i=1}^n \pout_i\,,
    \hspace{1cm}
    \pout = \bigcup_{i=0}^n \pout_i \setminus \pin_0
  \end{equation}
  and $\goesto$ is the minimal transition relation inductively defined by
  the derivation rule
  \begin{equation}
    \label{eq:composition:transitions}
    \renewcommand{\arraystretch}{1.5}
    \derrule[2]{
      \forall i \in [0,n],\, (a \cap P_i \neq \emptyset \implies 
        q_i \longgoesto[a \cap P_i] q_i') &
      \forall i \in [0,n],\, (a \cap P_i = \emptyset \implies q_i = q_i') 
    }{
      q_0 q_1 \dots q_n \longgoesto[a \cap P] q_0' q_1' \dots q_n'
    }\,,
  \end{equation}
  where, in the conclusion, $P \bydef{=} \pin \cup \pout$, with $\pin$ and
  $\pout$ defined in \eq{composition:ports}.
\end{definition}

\begin{example}[BIP interaction models]
  \label{ex:bip}
  Let $B_i = (Q_i, \emptyset, \pout_i, \goesto)$, for $i \in [1,n]$, be a
  set of behaviours with pairwise disjoint ports.  Consider $P =
  \bigcup_{i=1}^n \pout_i$ and let $\gamma \subseteq 2^P$ be a BIP
  interaction model \cite{BliSif07-acp-emsoft}.  The conversation $C_\gamma
  = \Big(\{*\}, P, \gamma, \Setdef{* \longgoesto[a\cup\{a\}] *}{a \in
    \gamma}\Big)$ realises the same coordination as the one imposed by
  $\gamma$, \ie $C_\gamma(B_1, \dots, B_n) = \gamma(B_1, \dots, B_n)$,
  where the latter is the composed component obtained by applying the
  interaction model $\gamma$ to behaviours $B_1, \dots, B_n$ with the
  operational semantics defined in \cite{BliSif07-acp-emsoft}.
\end{example}

\begin{definition}
  \label{defn:composition:parallel}
  Let $C_j = (Q_j, \pin_j, \pout_j, \goesto)$, for $j = 1,2$, be two
  independent conversations.  Their \emph{(true concurrency) parallel
    composition} is the conversation $C_1 \parallel C_2 \bydef{=} (Q, \pin,
  \pout, \goesto)$, where $Q = Q_1 \times Q_2$, $\pin = \pin_1 \cup
  \pin_2$, $\pout = \pout_1 \cup \pout_2$ and $\goesto$ is the minimal
  transition relation inductively defined by the following three derivation
  rules, with $a \subseteq P_1 \cup P_2$ ($P_i \bydef{=} \pin_i \cup
  \pout_i$, for $i=1,2$).
  \begin{equation}
    \label{eq:parallel}
    \derrule{q_1 \longgoesto[a] q_1'}{q_1 q_2 \longgoesto[a] q_1' q_2}\,,
    \hspace{1cm}
    \derrule[2]{
      q_1 \longgoesto[a \cap P_1] q_1' & 
      q_2 \longgoesto[a \cap P_2] q_2' 
    }{
      q_1 q_2 \longgoesto[a] q_1' q_2
    }\,,
    \hspace{1cm}
    \derrule{q_2 \longgoesto[a] q_2'}{q_1 q_2 \longgoesto[a] q_1 q_2'}\,.
  \end{equation}
\end{definition}

Clearly, parallel composition of independent conversations is associative.
Hence, this definition can be straightforwardly extended to an independent
set of conversations of any finite cardinality.  This also allows an
extension of the composition operator in \defn{composition} to
conversations applied in parallel to the same set of behaviours.

\begin{definition}
  \label{defn:composition:multi}
  Let $\setdef{T_i}{i \in [1,n]}$ be an independent set of RG-LTS and
  $\setdef{C_j}{j \in [1,m]}$ an independent set of conversations, such
  that $\setdef{T_i}{i \in [1,n]} \cup \setdef{C_j}{j \in [1,m]}$ is
  1-composable.  We define
  \[
    (C_1, \dots, C_m)(T_1, \dots, T_n) 
    \bydef{=} 
    \Big(\parallel(C_1, \dots, C_m)\Big)(T_1, \dots, T_n)\,.
  \]
\end{definition}

Clearly, thus defined composition can be applied hierarchically in a
straightforward manner.  

Notice that, in the context of \defn{composition:multi}, if all $T_i$ are
behaviours and every conversation grant port is a behaviour request port,
$(C_1, \dots, C_m)(T_1, \dots, T_n)$ is a behaviour.

\begin{example}[Network sort]
  \label{ex:network}
  ...
\end{example}

Composition defined by hierarchical application of the operator in
\defn{composition:multi} allows simultaneous execution of interactions from
any number of top-level conversations.  However, mutual exclusion among
interactions can be ensured by applying a single top-level arbiter
conversation.  Such arbiter conversation can enforce simple
non-deterministic mutual exclusion, as in \ex{bip}, or any complex
scheduling protocol, \eg Round-Robin.

Notice that when, as in \ex{network}, the system is given as a directed
acyclic graph (DAG), it can be grouped into a hierarchical system in
different ways.  The following proposition shows that operational semantics
of such systems does not depend on the grouping.

\begin{proposition}
  ...
\end{proposition}

% *********************************************************************

\subsection{Data Transfer}
\label{sec:data}

We now extend the definitions of \secn{external:model} to incorporate data.
For simplicity, we assume that a universal data domain $D$ is given.


% *********************************************************************

\subsection{Expressiveness Results}
\label{sec:expressiveness}


Types of conversations:
\begin{itemize}
\item Linear (finite or cyclic) vs.\ branching.  Branching can be
  \begin{itemize}
  \item deterministic
  \item non-deterministic
  \end{itemize}
\item With or without memory (``unbounded'' data values)
\item One label vs.\ multiple ``send'' labels for interactions composing
  the conversation
\end{itemize}

Conversations (deterministic or not) are strictly more expressive than
interactions.

Deterministic and non-deterministic conversations without memory are
equivalent.

Non-deterministic conversations with memory are strictly more expressive
than deterministic conversations with memory
