% *********************************************************************
% *********************************************************************

\section{Formal Model}
\label{sec:formal}

% *********************************************************************

\subsection{Grant-Request Labelled Transition Systems}
\label{sec:model}

Intuitively, we consider systems built from a number of components, each
modelled by a labelled transition system (LTS).  In order to perform a
transition, a basic component, which we call \emph{behaviour}, issues a
number of requests, which are defined by the transition label and can be
granted by higher-level components, which we call \emph{conversations}.
Thus, in general, a complete transition label consists of two sets of {\em
  ports}, representing requests the transition must grant and those it must
obtain in order to be executed.  Below we present a formal model specifying
such behaviour as well as the implied composition operator.

\begin{definition}[\compmodel{} and interactions]
  \label{defn:lts}
  A \emph{Grant-Request Labelled Transition System} (\compmodel{}) is a
  quadruple $(Q,\pin,\pout,\goesto{})$, where $Q$ is a set of states, $\pin
  \cap \pout = \emptyset$ are finite sets of respectively \emph{grant} and
  \emph{request} ports and $\goesto{} \subseteq Q \times 2^P \times Q$
  (with $P = \pin \cup \pout$) is a transition relation.  Transitions are
  labelled by \emph{interactions} $\emptyset \neq a \subseteq P$.  We
  define $q \goesto{a} q' \bydef{=}\Big((q, a, q') \in \goesto{}\Big)$ and
  $q \goesto{a}\, \bydef{=}\Big(\exists q' \in Q: q \goesto{a} q'\Big)$.
  Here and below, we skip the index on $\goesto{}_i$ since it is always clear
  from the context.
\end{definition}

\begin{notation}
  \label{ntn:lts}
  For a \compmodel{} with the sets of grant and request ports respectively
  $\grant{X}$ and $\request{X}$ (here, $X$ can be $P$, $P_i$, \etc[]) we
  will always denote $X \bydef{=} \grant{X} \cup \request{X}$.  Conversely,
  for any object $X$ that can be viewed as a mapping $X : P \rightarrow \_$
  (\eg an interaction $a$ can be viewed as the characteristic function $a :
  P \rightarrow \{\true,\false\}$), we denote by $\grant{X}$ and
  $\request{X}$ the respective restrictions of $X$ to $\grant{P}$ and
  $\request{P}$.
\end{notation}

\begin{definition}[System]
  \label{defn:system}
  Let $S = \setdef{T_i = (Q_i, \pin_i, \pout_i, \goesto{})}{i \in [1,n]}$
  be a finite set of \compmodel{} and denote $\pin \bydef{=}
  \bigcup_{i=1}^n \pin_i$ and $\pout \bydef{=} \bigcup_{i=1}^n \pout_i$.
  $S$ is a \emph{system} iff the sets of request ports of all the
  \emph{components} are pairwise disjoint, \ie $\forall i \neq j,\ \pout_i
  \cap \pout_j = \emptyset$.

  A system is \emph{closed} if $\pin = \pout$.  Otherwise it is
  \emph{open}.  
\end{definition}

%% To simplify the presentation, below we postulate that for any $q$, the
%% predicate $q \goesto{\emptyset} q$ is true, whenever it appears in the
%% premises of an SOS rule.

\begin{definition}[System composition]
  \label{defn:composition}
  Consider a system $S = \Setdef{T_i = (Q_i, \pin_i, \pout_i, \goesto{})}{i
    \in [1,n]}$.  The composition of $S$ is given by a \compmodel{}
  $\compose[S] \bydef{=} (Q, \pin, \pout, \goesto{})$, where $Q =
  \prod_{i=1}^n Q_i$, $\pin = \bigcup_{i=1}^n \pin_i$, $\pout =
  \bigcup_{i=1}^n \pout_i$
  %% \begin{equation}
  %%   \label{eq:composition:parts}
  %%   Q = \prod_{i=1}^n Q_i\,,
  %%   \hspace{1cm}
  %%   \pin = \bigcup_{i=1}^n \pin_i %\setminus \bigcup_{i=1}^n \pout_i\,,
  %%   \hspace{1cm}
  %%   \pout = \bigcup_{i=1}^n \pout_i %\setminus \bigcup_{i=1}^n \pin_i 
  %% \end{equation}
  and $\goesto{}$ is the minimal transition relation inductively defined by
  the derivation rule
  \begin{equation}
    \label{eq:composition:transitions}
    \renewcommand{\arraystretch}{1.5}
    \derrule[3]{
      a \subseteq \bigcup_{i=1}^n P_i &
      \forall i \in [1,n],\, \Big(a \cap P_i \neq \emptyset \implies 
        q_i \goesto{a \cap P_i} q_i'\Big) &
      \forall i \in [1,n],\, \Big(a \cap P_i = \emptyset \implies 
        q_i = q_i'\Big) 
    }{
      q_1 \dots q_n \goesto{a \cap P} q_1' \dots q_n'
    }\,,
  \end{equation}
  with $P_i$ and $P$ as in \ntn{lts}.
\end{definition}

\begin{proposition}
  \label{prop:associativity}
  Composition operator $\compose[]$ in \defn{composition} is associative.
\end{proposition}

\begin{definition}[Behaviour and conversation]
  \label{defn:components}
  A \compmodel{} $(Q, \pin, \pout, \goesto{})$ is
  \begin{itemize}
  \item a \emph{behaviour} if it is grant-closed, \ie $\pin \subseteq
    \pout$;
  \item a \emph{conversation} if, for any transition $q \goesto{a} q'$, both
    $a \cap \pin \neq \emptyset$ and $a \cap \pout \neq \emptyset$.
  \end{itemize}
\end{definition}

Conversations represent sequences of interactions. They are used to compose
behaviours.\sidenote{Is this a meaningful restriction?} Below we will
only consider systems, whereof each component is either a behaviour or a
conversation.

\begin{example}[BIP interaction models]
  \label{ex:bip}
  Let $B_i = (Q_i, \emptyset, \pout_i, \goesto{})$, for $i \in [1,n]$, be a
  set of behaviours with pairwise disjoint ports.  Consider $P =
  \bigcup_{i=1}^n \pout_i$ and let $\gamma \subseteq 2^P$ be a BIP
  interaction model \cite{BliSif07-acp-emsoft}.  The conversation $C_\gamma
  = \Big(\{*\}, P, \gamma, \Setdef{* \goesto{a\cup\{a\}} *}{a \in
    \gamma}\Big)$ realises the same coordination as the one imposed by
  $\gamma$, \ie $C_\gamma(B_1, \dots, B_n) \bydef{=} \compose[\{C_\gamma,
    B_1, \dots, B_n\}] = \gamma(B_1, \dots, B_n)$, where the latter is the
  composed component obtained by applying the interaction model $\gamma$ to
  behaviours $B_1, \dots, B_n$ with the operational semantics defined in
  \cite{BliSif07-acp-emsoft}.
\end{example}

\begin{example}[Network sort]
  \label{ex:network}
  ...
\end{example}

\begin{definition}[Topologies]
  \label{defn:topology}
  The \emph{topology} of a system $S$ is a directed graph $\topo{S} = (S,
  E)$, having the components of the system as vertices and the set of edges
  $E = \setdef{(T_i,T_j)}{\pout_i \cap \pin_j \neq \emptyset}$.  In other
  words, there is an edge from $T_i$ to $T_j$ if some of the requests of
  the former can be granted by the latter.  A component $T \in S$ is called
  \emph{top-level}, if the corresponding node in $\topo{S}$ does not have
  any outgoing edges (\ie component is a sink in $\topo{S}$).

  If $\topo{S}$ is a directed acyclic graph (DAG) then $S$ is a
  \emph{flow-system}.  If $\topo{S}$ is disconnected, \ie $E = \emptyset$,
  the system is \emph{independent}.
\end{definition}

Composition defined by the operational semantics of systems allows
simultaneous execution of interactions authorised by any number of
top-level conversations.  However, mutual exclusion can be ensured by
applying a single top-level arbiter conversation.  Such arbiter
conversation can enforce simple non-deterministic mutual exclusion, as in
\ex{bip}, or any complex scheduling protocol, \eg Round-Robin.

% *********************************************************************

\subsection{Component Model with Data}
\label{sec:data}

We now extend the definitions of \secn{model} to incorporate data.  To
avoid dealing with causality cycles, we limit ourselves to flow-systems,
where causal data dependencies can be unambigously resolved.  Intuitively,
in each cycle, we propagate the data from behaviours upwards through all
relevant conversations.  At each component, this data can influence the
decision as to what transitions are enabled.  Finally, once a global
interaction has been choosen at the top level, the updated data is
propagated back to behaviours.  Below, we present this in a formal manner.

\begin{definition}[\datamodel{}]
  \label{defn:data}
  A \emph{\datamodel{}} is a tuple $T=(Q, \pin, \pout, \goesto{},
  \local{\data}, \data, \guard, \up, \down, \combine)$, where
  \begin{itemize}
  \item $(Q, \pin, \pout, \goesto{})$ is a \compmodel{},
  \item $\local{\data}$ is the \emph{local data domain},
  \item $\data = \setdef{\data[p]}{p \in P}$ is a set of \emph{data domains
    associated to ports} of the system,
  \item $\guard$ is the \emph{guard} mapping associating to each transition
    $t = q \goesto{a} q'$ a boolean guard $\guard(t): \local{\data} \times
    \data[\grant{a}] \rightarrow \{\true,\false\}$,
  \item $\up$ and $\down$ are \emph{data transfer} mappings associating to
    each transition $t = q \goesto{a} q'$ two transfer functions $\up(t):
    \local{\data} \times \data[\grant{a}] \rightarrow \local{\data} \times
    \data[\request{a}]$ and $\down(t): \local{\data} \times \data[P]
    \rightarrow \local{\data} \times \data[\grant{a}]$, where we denote,
    for a set of ports $X \subseteq P$, $\data[X] \bydef{=} \prod_{p \in X}
    \data[p]$.  $\up(t)$ propagates the data up from the grant ports to the
    request ports, whereas $\down(t)$ computes new values for the grant
    port variables.  Both functions can update the local variable.
  \item Finally, $\combine = \setdef{(\combine[p], \bot^p)}{p \in \pout}$,
    where $\combine[p]: (\data[p])^2 \rightarrow \data[p]$ is an
    associative \emph{aggregation function}, for combining several
    proposals for the value associated to $p$ into one, and $\bot^p$ is an
    \emph{identity element}, such that $\forall x \in \data[p],
    \combine[p](x, \bot^p) = \combine[p](\bot^p, x) = x$.
  \end{itemize}

  By abuse of notation, we write $q \goesto{a,\guard,\up,\down} q'$ for
  the transition $q \goesto{a} q' \in\, \goesto{}$ and the associated guard
  and data transfer functions.
\end{definition}

Notice\sidenote{not clear} that, in \defn{data}, all the data preserved
between two execution cycles is stored in the local variable.  Thus, during
the upward propagation, data associated to request ports are not taken into
account.  On the contrary, downward propagation takes into account all
data.  Precise operational semantics of a \datamodel{} is given by the
following definition.

The aggregation function allows us to define the downward phase of the data
transfer when one request port has several grant instances in other
components (see \secn{data:composition}).  Some examples of such
aggregation functions are $\mathbf{max}$, $+$, $\mathbf{first}(x_1, x_2) = x_1$.
Another important example is 
%
\[
\mathbf{ndc}(x_1, x_2) \bydef{=} 
\begin{cases}
  x_1, & \mbox{if } x_2 = \bot,\\
  x_2, & \mbox{if } x_1 = \bot,\\
  x_i, & \mbox{where $i \in \{1,2\}$ is chosen non-deterministically, 
    otherwise.}
\end{cases}
\] 
%
Although, strictly speaking, $\mathbf{ndc}()$ is not a function, it
posesses all the properties necessary in our context.

\begin{definition}[Component operational semantics with data]
  \label{defn:data:semantics}
  The operational semantics of a \datamodel{} $T = (Q, \local{\data},
  \data, \pin, \pout, \goesto{}, \guard, \up, \down, \combine)$ is given by
  an LTS $\undata{T} = (Q \times \local{\data}, 2^P \times \data[P] \times
  \data[P]$, \mbox{$\goesto{})$}, where a state $(q, v)$ consists of a
  control state of $T$ and the value $v \in \local{\data}$; $\goesto{}$ is
  the minimal transition relation inductively defined by the following
  rule:
  \begin{equation}
    \label{eq:data:semantics}
    \renewcommand{\arraystretch}{1.5}
    \derrule[4]{
      q \goesto{a,\guard,\up,\down} q'
      &
      \guard(v, \grant{\portdata{v}}) = \true
      &
      \request{\portdata{v}} = 
      \request{\up}(v, \grant{\portdata{v}})
      &
      (v', \grant{\portdata{v'}}) =
      \down\Big(
      \local{\up}(v, \grant{\portdata{v}}),
      \request{\portdata{v'}}, 
      \grant{\portdata{v}}
      \Big)
    }{
      (q, v)
      \goesto[\portdata{v}:\portdata{v'}]{a}
      (q', v')
    }\,,
  \end{equation}
  where $\local{\up}$, $\request{\up}$ and $\local{\down}$ are the
  corresponding components of $\up$ and $\down$ functions respectively;
  $\portdata{v}, \portdata{v'} \in \data[P]$ are \emph{data values
    associated to ports at the upward and downward data transfer phases}
  respectively.  Notice that, although ports can be shared among
  components, different values can be associated to the same port in
  different components.  This allows us to define $\grant{\portdata{v'}}$
  in \eq{data:semantics} independently of how this component is to be
  composed.
\end{definition}

\begin{note}
  \label{rem:io}
  For the values $\portdata{v}$ and $\portdata{v'}$, in
  \eq{data:semantics}, it is important to notice the difference with the
  input/output dichotomy.  Indeed, the component \emph{input} is the pair
  $(\grant{\portdata{v}}, \request{\portdata{v'}})$, whereas its
  \emph{output} is the pair $(\grant{\portdata{v'}},
  \request{\portdata{v}})$.  Hence, the only constraint on
  $(\grant{\portdata{v}}, \request{\portdata{v'}})$ is the guard in the
  second premise.
\end{note}
%% It is important to notice that, in \eq{data:semantics}, the request
%% component of the upward transfer function $\up$ is discarded and both the
%% downward transfer and destination state rely on the unrestricted value
%% $\portdata{v}$. This reflects the fact that, as discussed above, at each
%% execution cycle the data associated to all the ports is discarded.  In a
%% closed system, there are two phases: in the upwards transfer phase, the
%% values associated to the grant ports are initialised by the corresponding
%% request ports in the components one level down in the topology and
%% vice-versa, in the downwards transfer phase.

% *********************************************************************

\subsection{Composition of Components with Data}
\label{sec:data:composition}

\begin{definition}[Data coherency]
  \label{defn:data:coherency}
  Let \[S = \Setdef{T_i = (Q_i, \pin_i, \pout_i, \goesto{},
    \local{\data}_i, \data_i, \guard_i, \up_i, \down_i, \combine_i)}{i \in
    [1,n]}\] be a set of \datamodel{}.  We call $S$ \emph{data-coherent}
  if, for each port belonging to several components, the corresponding data
  domains coincide:
  \begin{equation}
    \label{eq:data:coherency}
    \forall p \in P,\ \forall i,j \in [1,n],\ 
    \Big(
    p \in \pout_i \cap \pin_j \implies \data[p]_i = \data[p]_j
    \Big)\,,
  \end{equation}
  where $P \bydef{=} \bigcup_{i=1}^n P_i$.
\end{definition}

Recall that request port sets of different components in a system are
disjoint, hence only one aggregation function is defined for each request
port.  Thus, there is no need to take aggregation functions into account in
the definition of coherency.

\begin{definition}[Composition of \datamodel{}]
  \label{defn:data:composition}
  Let $T_i = (Q_i, \pin_i, \pout_i, \goesto{}, \local{\data}_i, \data_i,
  \guard_i, \up_i, \down_i, \combine_i)$, for $i=1,2$, be two data-coherent
  \datamodel{}.  Their parallel composition is a \datamodel{} $T_1
  \parallel T_2 \bydef{=} (Q, \pin, \pout, \goesto{}, \local{\data}, \data,
  \guard, \up, \down, \combine)$, where $(Q, \pin, \pout, \goesto{}) =
  (Q_1, \pin_1, \pout_1, \goesto{}) \parallel (Q_2, \pin_2, \pout_2,
  \goesto{})$ (see \defn{composition}), $\local{\data} = \local{\data}_1
  \times \local{\data}_2$, $\data = \data_1 \cup \data_2$, $\combine =
  \combine_1 \cup \combine_2$ (here $+$ is the function coproduct) and, for
  each transition $t = (q \goesto{a} q')$, putting $a_i = a \cap P_i$, for
  $i=1,2$,
  \[
    \guard(t)(v_1, v_2, \grant{\portdata{v}}_1,\grant{\portdata{v}}_1)\ 
    \bydef{=}\
    \guard(t_1)(v_1, \grant{\portdata{v}}_1) \land
    \guard(t_2)(v_2, \grant{\portdata{v}}_2) \land
    \Big(\forall p \in \pin_2 \cap \pout_1,\,
    \portdata{v}^p_2 = \up^p(v_1, \grant{\portdata{v}}_1)\Big)\,;
  \]
  the functions $\up$ and $\down$ are computed as follows (\cf
  \rem{io}):
  \begin{enumerate}
  \item \label{up1} Applying $\up_1$ to the locally stored value $v_1$ and
    $\grant{\portdata{v}}_1$, in $T_1$, compute the intermediate locally
    stored value $\tilde{v}_1$ and data values
    $\request{\portdata{\tilde{v}}}_1$ associated to the request ports in
    $\request{a}_1$;
  \item \label{transfer1} For all shared ports $p \in \pin_2 \cap \pout_1$,
    assign the corresponding intermediate values $\portdata{\tilde{v}}^p_2
    = \portdata{\tilde{v}}^p_1$, ports $p \in \pin_2 \setminus \pout_1$
    stay unchanged: $\portdata{\tilde{v}}^p_2 = \portdata{v}^p_2$;
  \item \label{up2} Applying $\up_2$ to the locally stored value $v_2$ and
    $\grant{\portdata{\tilde{v}}}_2$, in $T_2$, compute the intermediate
    locally stored value $\tilde{v}_2$ and the data values
    $\request{\portdata{v}}_2$ associated to the request ports in
    $\request{a}_2$;
  \item \label{down2} Applying $\down_2$ to $\tilde{v}_2$ in $T_2$ and
    $\request{\portdata{v'}}_2$, compute the final locally stored value
    $v'_2$;  and the data values $\grant{\portdata{v'}}_2$ associated to the
    grant ports in $\grant{a}_2$;
  \item \label{transfer2} For all ports $p \in \pin_2 \cap \pout_1$,
    compute the corresponding values by assigning
    $\portdata{v'}^p_1 := \combine[p]_1(\portdata{v'}^p_2, \portdata{v'}^p_1)$;
  \item \label{down1} Applying $\down_1$ to $\tilde{v}_1$ in $T_1$ and
    $\request{\portdata{v'}}_1$, compute the final locally stored value
    $v'_1$;  and the data values $\grant{\portdata{v'}}_1$ associated to the
    grant ports in $\grant{a}_1$;
  \item \label{final} Function values to be defined are then given by the
    two equations
    \[
    \up(v_1, v_2, \grant{\portdata{v}}_1, \grant{\portdata{v}}_2)
    \bydef{=}
    (\tilde{v}_1, \tilde{v}_2, 
    \request{\portdata{v}}_1, \request{\portdata{v}}_2)\,,
    \hspace{1cm}
    \down(v_1, v_2, 
      \request{\portdata{v'}}_1, 
      \grant{\portdata{v}}_1, 
      \request{\portdata{v'}}_2, 
      \grant{\portdata{v}}_2, 
    )
    \bydef{=}
    (v'_1, v'_2, 
      \grant{\portdata{v'}}_1, 
      \grant{\portdata{v'}}_2
    )\,.
    \]
  \end{enumerate}
  The only data preserved between successive computations above are the
  locally stored values $v_1$ and $v_2$, which are also parameters of both
  $\up$ and $\down$.  Also, it is clear that $\up$ and $\down$ are defined
  whenever $\up_i$ and $\down_i$ are defined for both $i = 1,2$.  Hence
  $\up$ and $\down$, defined by the above procedure, are, indeed,
  functions.
\end{definition}

\begin{note}
  \label{rem:transfer}
  In the above definition, it is important to observe that during the upward
  data transfer the values $\portdata{v}_2$ associated to the ports realizing
  the connection between $T_1$ and $T_2$ are discarded and replaced by the
  intermediate values computed at step \ref{transfer1}.  Therefore, even
  though these values are listed as parameters of the $\up$ function, they do
  not affect its computed value.  On the contrary, during the downward phase
  (step \ref{transfer2}), the corresponding values $\portdata{v'}_1$ are not
  overwritten, but combined with the values computed at step \ref{down2}.
\end{note}

This definition can be extended to arbitrary flow-systems in a
straightforward manner.

\begin{proposition}
  \label{prop:data:associativity}
  Composition operator $\parallel$ in \defn{data:composition} is
  associative.
\end{proposition}
%
\begin{proof}[Sketch of the proof]
  \prop{associativity} states that parallel composition of the underlying
  \compmodel{} is associative.  Hence, we only have to show that the
  definition of data transfer and guard functions is consistent.  

  All computation steps \ref{up1}--\ref{final} in \defn{data:composition}
  are performed locally.  Hence, to prove associativity, it is sufficient
  to show that the input values provided to the functions involved are
  independent of the computation path.

  Since the topology of any flow-system is a DAG, for each request port in
  the system, there is a unique maximal path leading to this port in the
  underlying topology.  This path unambigously defines the calculation of
  the data value associated to this port during the upward data transfer
  phase.  As an immediate consequence, all the guard valuations are also
  unambigously defined.

  For the downward transfer phase, the values associated to the grant ports
  are defined by the combination of all paths in the inverted topology,
  leading to this port from the top-level conversations participating in
  the interaction.  However, the only conflicts possible during this
  computation appear at steps \ref{transfer2} in \defn{data:composition}.
  Since all aggregation functions are associative, this does not impact any
  of the intermediate values.
\end{proof}

% *********************************************************************

\subsection{Modularity}
\label{sec:modularity}

In the previous sections, we have defined a component model and operational
semantics, combining \compmodel{} with data management and transfer
(\defn{data}).  In this section, we show how such components can be bundled
together to be used as \emph{modules} in larger systems.  A common
intuition, when a flow-system is built to be used as a subsystem of a
larger one, is that the ports of the subsystem are the non-connected ports
of its components.  

\begin{definition}[Internal and external ports]
  \label{defn:port:types}
  Let $S$ be a system with the sets of request and grant ports $\pin$ and
  $\pout$ respectively.  We call the ports $p \in \pin \cap \pout$
  \emph{internal} and $p \in \pin \Delta \pout$ \emph{external}.
\end{definition}

\begin{definition}[Hiding of ports]
  \label{defn:data:hiding}
  Let \[S = \Setdef{T_i = (Q_i, \pin_i, \pout_i, \goesto{},
    \local{\data}_i, \data_i, \guard_i, \up_i, \down_i, \combine_i)}{i \in
    [1,n]}\] be a data-coherent flow-system and $\compose[S] = (Q, \pin,
  \pout, \goesto{}, \local{\data}, \data, \guard, \up, \down, \combine)$ be
  the corresponding \datamodel{} as defined in \defn{data:composition}.
  For any subset of ports $X \subseteq P$, the application to $\compose[S]$
  of the \emph{hiding operator} $\hide[X]{\cdot}$ is given by the
  \datamodel{} $\hide[X]{\compose[S]} = (Q, \pin \setminus X, \pout
  \setminus X, \goesto{}, \local{\data}, \setdef{\data[p]}{p \in P
    \setminus X}, \guard, \up, \down')$.  Here, $Q$, $\local{\data}$,
  $\goesto{}$, $\guard$ and $\up$ are the same as in $\compose[S]$.  The
  function $\down'$ is defined as follows
  \begin{equation}
    \label{eq:data:hiding}
    \down'\Big(v_1, \dots, v_n, 
      (\portdata{\tilde{v}}^p)_{p \in \pout \setminus X}, 
      (\portdata{v}^p)_{p \in \pin \setminus X}
    \Big) 
    \bydef{=} 
    \down\Big(v_1, \dots, v_n, 
      (\portdata{\tilde{v}}^p)_{p \in \pout \setminus X}, 
      (\bot^p)_{p \in \pout \cap X},
      (\portdata{v}^p)_{p \in \pin \setminus X}
    \Big)\,.
  \end{equation}

  Recall (\rem{transfer}) that the values $\portdata{v}^p$, for $p \in \pin
  \cap X$ do not affect the value of $\down()$.  Hence, there is no need to
  specify them in \eq{data:hiding}.

  We denote $\hide[int]{\compose{S}} \bydef{=} \hide[\pin \cap
    \pout]{\compose{S}}$ the \datamodel{} obtained by hiding all internal
  ports in the composed system $S$.
\end{definition}

\begin{definition}[Flow-system operational semantics with data]
  \label{defn:flow:semantics}
  Let \[S = \Setdef{T_i = (Q_i, \pin_i, \pout_i, \goesto{},
    \local{\data}_i, \data_i, \guard_i, \up_i, \down_i, \combine_i)}{i \in
    [1,n]}\] be a data-coherent flow-system.  Its operational semantics is
  given by an LTS $\undata{S} = (Q \times \local{\data}, 2^P \times
  \data[P] \times \data[P], \goesto{})$, with
  \begin{equation}
    \label{eq:flow:parts}
    Q = \prod_{i=1}^n Q_i\,,
    \hspace{1cm}
    P = \left(\bigcup_{i=1}^n \pin_i\right) 
      \Delta 
      \left(\bigcup_{i=1}^n \pout_i\right)\,,
    \hspace{1cm}
    \local{\data} = \prod_{i=1}^n \local{\data}_i\,,
    \hspace{1cm}
    \data = \Setdef{\data[p]}{p \in P}\,,
  \end{equation}
  where $\Delta$ is the symmetric difference; \goesto{} is the minimal
  transition relation inductively defined by the derivation rule
  \begin{equation}
    \label{eq:flow:semantics}
    \renewcommand{\arraystretch}{1.5}
    \derrule[3]{
      a \subseteq P
      &
      \forall i \in [1,n],\,
      (q_i,v_i) 
      \goesto[\portdata{v}_i:\portdata{\tilde{v}}_i]{a \cap P_i}
      (q_i', v_i')
      &
      \forall i,\,
      \forall p \in \pout_i \cap \pin,
      \begin{cases}
      \forall j,\, p \in \pin_j \Rightarrow 
      \portdata{v}_i^p = \portdata{v}_j^p\\
      %% \exists j \in [1,n]: p \in \pin_j \land 
      \portdata{\tilde{v}}^p_i = 
      \combine[p]_i \Big((\portdata{\tilde{v}}^p_j)_{p \in \pin_j}\Big)\\
      \end{cases}
    }{      
      (q_1\dots q_n, v_1\dots v_n)
      \goesto[\portdata{v}:\portdata{\tilde{v}}]{a}
      (q_1'\dots q_n', v_1'\dots v_n')
    }\,,
  \end{equation}
  where $\portdata{v}^p = \portdata{v}^p_i$ and $\portdata{\tilde{v}}^p =
  \portdata{\tilde{v}}^p_i$, for all $p \in P \cap P_i$, and the second
  premise is a transition in $\undata{T_i}$.
\end{definition}

The third premise in \eq{flow:semantics} states that, during the upward
transfer phase, the data is propagated in a \emph{broadcast} manner, \ie
the same value is transferred from a request instance of a port to
\emph{all} corresponding grant instances; during the downward transfer
phase, request values are assigned aggregations of the corresponding grant
values. 

%% \begin{lemma}
%%   For two flow-systems $S_1$ and $S_2$ that only share external ports and
%%   such that $S_1 \cup S_2$ is data-coherent, holds the property
%%   \begin{equation}
%%     \hide[int]{\hide[int]{\compose{S_1}} \parallel \hide[int]{\compose{S_2}}}
%%     =
%%     \hide[int]{\compose{(S_1 \cup S_2)}}\,,
%%   \end{equation}
%%   where \hide[int]{\cdot} is the operator hiding all internal ports of the
%%   operand system.
%% \end{lemma}

\begin{theorem}
  \label{thm:compositionality}
  Operational semantics of the flow-systems with data is compositional. In
  other words, for any  data-coherent system of \datamodel{} $S$,
  holds
  \begin{equation}
    \label{eq:compositionality}
    \undata{\hide[int]{\compose[S]}} = \undata{S}\,,
  \end{equation}
  where, in the left-hand side, $\undata{\cdot}$ is the semantic function
  on \datamodel{} introduced in \defn{data:semantics}, whereas, in the
  right-hand side, $\undata{\cdot}$ is the semantic function on
  flow-systems introduced in \defn{flow:semantics}.
\end{theorem}
%
\begin{proof}[Sketch of the proof]
  Let $S = \setdef{(Q_i, \pin_i, \pout_i, \goesto{}, \local{\data}_i,
    \data_i, \guard_i, \up_i, \down_i, \combine_i)}{i \in [1,n]}$.
  Clearly, all elements composing the state space and the label set of the
  LTS on both sides of \eq{compositionality} coincide.  Therefore, we only
  have to show that so do the corresponding transition relations.  By
  comparing \eq{data:hiding} and \eq{flow:semantics}, we conclude that this
  follows from the requirement that, for any $p \in P$ and any $x \in
  \data[p]$, $\combine[p](x, \bot^p) = x$.
\end{proof}

% *********************************************************************

\subsection*{Work In Progress Notes}

In the definition of \datamodel{} (\defn{data}), we can add another level
of inderection, by replacing the constant identity element $\bot^p$ by an
associative function $\bot^p: (\data[p])^2 \rightarrow \data[p]$.  This
would allow us to consider yet another aggregation function
\[
  \mathbf{equals}(x_1,x_2) = \bot(x_1,x_2) = 
  \begin{cases}
    x_1, & \mbox{if } x_1 = x_2,\\
    error, & \mbox{if } x_1 \neq x_2.
  \end{cases}
\]
For other examples, the $\bot^p$ function is constant.  

However, I think this would only obscure the formalisation.

% *********************************************************************

\subsection{Expressiveness Results}
\label{sec:expressiveness}

{\em Expressiveness results should probably go into a separate section at
  the end, after the implementation details and the use case.}

Types of conversations:
\begin{itemize}
\item Linear (finite or cyclic) vs.\ branching.  Branching can be
  \begin{itemize}
  \item deterministic
  \item non-deterministic
  \end{itemize}
\item With or without memory (``unbounded'' data values)
\item One label vs.\ multiple ``send'' labels for interactions composing
  the conversation
\end{itemize}

Conversations (deterministic or not) are strictly more expressive than
interactions.

Deterministic and non-deterministic conversations without memory are
equivalent.

Non-deterministic conversations with memory are strictly more expressive
than deterministic conversations with memory

