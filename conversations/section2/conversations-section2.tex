% *********************************************************************
% *********************************************************************

\section{Formal Model}
\label{sec:formal}

% *********************************************************************

\subsection{Grant-Request Labelled Transition Systems}
\label{sec:model}

Intuitively, we consider systems built from a number of components, each
modelled by a labelled transition system (LTS).  In order to perform a
transition, a basic component, which we call {\em behaviour}, issues a
number of requests, which are defined by the transition label and can be
granted by higher-level components, which we call {\em conversations}.
Thus, in general, a complete transition label consists of two sets of {\em
  ports}, representing requests the transition must grant and those it must
obtain in order to be executed.  Below we present a formal model specifying
such behaviour as well as the implied composition operator.

\begin{definition}[\compmodel{} and interactions]
\label{defn:lts}
A \emph{Grant-Request Labelled Transition System} (\compmodel{}) is a
quadruple $(Q,\pin,\pout,\goesto[])$, where $Q$ is a finite set of states,
$\pin \cap \pout = \emptyset$ are finite sets of respectively {\em grant}
and {\em request} ports and $\goesto[] \subseteq Q \times 2^P \times Q$
(with $P = \pin \cup \pout$) is a transition relation.  Transitions are
labelled by \emph{interactions} $\emptyset \neq a \subseteq P$.  We define
$q \goesto[a] q' \bydef{=}\Big((q, a, q') \in \goesto[]\Big)$ and $q
\goesto[a]\, \bydef{=}\Big(\exists q' \in Q: q \goesto[a] q'\Big)$.
\end{definition}

\begin{notation}
  \label{ntn:lts}
  In \defn{lts} and below, we skip the index on $\goesto_i$ since it is
  always clear from the context.  For a \compmodel{} with the sets of grant
  and request ports respectively $\grant{X}$ and $\request{X}$ (here, $X$
  can be $P$, $P_i$, \etc[]) we will always denote $X \bydef{=} \grant{X}
  \cup \request{X}$.
\end{notation}

\begin{definition}[System]
  \label{defn:system}
  Let $S = \setdef{T_i = (Q_i, \pin_i, \pout_i, \goesto)}{i \in [1,n]}$ be
  a finite set of \compmodel{} and denote $\pin \bydef{=} \bigcup_{i=1}^n
  \pin_i$ and $\pout \bydef{=} \bigcup_{i=1}^n \pout_i$.  $S$ is a {\em
    system} iff the following two conditions hold:
  \begin{enumerate}
  \item the sets of request ports of all the {\em components} are pairwise
    disjoint, \ie $\forall i \neq j,\ \pout_i \cap \pout_j = \emptyset$;
  \item any grant port belonging to more than one component is also a
    request port of some component, \ie $\forall p, \left(\exists i \neq 
    j: p \in \pin_i \cap \pin_j \implies p \in \pout\right)$.
  \end{enumerate}

  A system is {\em closed} if $\pin \subseteq \pout$.  Otherwise it is {\em
    open}.
\end{definition}

To simplify the presentation, below we postulate that for any $q$, the
predicate $q \goesto[\emptyset] q$ is true, whenever it appears in the
premises of an SOS rule.

\begin{definition}[Operational semantics of a system]
  \label{defn:composition}
  The operational semantics of a system $S = \Setdef{T_i = (Q_i, \pin_i,
  \pout_i, \goesto)}{i \in [1,n]}$ is given by a \compmodel{} $\compose[S]
  \bydef{=} (Q, \pin, \pout, \goesto)$, where
  \begin{equation}
    \label{eq:composition:ports}
    Q = \prod_{i=1}^n Q_i\,,
    \hspace{1cm}
    \pin = \bigcup_{i=1}^n \pin_i \setminus \bigcup_{i=1}^n \pout_i\,,
    \hspace{1cm}
    \pout = \bigcup_{i=1}^n \pout_i \setminus \bigcup_{i=1}^n \pin_i 
  \end{equation}
  and $\goesto$ is the minimal transition relation inductively defined by
  the derivation rule
  \begin{equation}
    \label{eq:composition:transitions}
    \renewcommand{\arraystretch}{1.5}
    \derrule[3]{
      a \subseteq \bigcup_{i=1}^n P_i &
      \forall i \in [1,n],\, (a \cap P_i \neq \emptyset \implies 
        q_i \longgoesto[a \cap P_i] q_i') &
      \forall i \in [1,n],\, (a \cap P_i = \emptyset \implies q_i = q_i') 
    }{
      q_1 \dots q_n \longgoesto[a \cap P] q_1' \dots q_n'
    }\,,
  \end{equation}
  with $P_i$ and $P$ as in \ntn{lts}.
\end{definition}

\begin{proposition}
  \label{prop:associativity}
  Composition operator $\compose$ in \defn{composition} is associative.
\end{proposition}

\begin{definition}[Behaviour and conversation]
  \label{defn:components}
  A \compmodel{} $(Q, \pin, \pout, \goesto)$ is
  \begin{itemize}
  \item a {\em behaviour} if $\pin  = \emptyset$;
  \item a \emph{conversation} if, for any transition $q \goesto[a] q'$, both
    $a \cap \pin \neq \emptyset$ and $a \cap \pout \neq \emptyset$.
  \end{itemize}
\end{definition}

Conversations represent sequences of interactions. They are used to compose
behaviours.  Notice that operational semantics of a closed system is a
behaviour.  Below we will only consider systems, whereof each component is
either a behaviour or a conversation.

\begin{example}[BIP interaction models]
  \label{ex:bip}
  Let $B_i = (Q_i, \emptyset, \pout_i, \goesto)$, for $i \in [1,n]$, be a
  set of behaviours with pairwise disjoint ports.  Consider $P =
  \bigcup_{i=1}^n \pout_i$ and let $\gamma \subseteq 2^P$ be a BIP
  interaction model \cite{BliSif07-acp-emsoft}.  The conversation $C_\gamma
  = \Big(\{*\}, P, \gamma, \Setdef{* \longgoesto[a\cup\{a\}] *}{a \in
  \gamma}\Big)$ realises the same coordination as the one imposed by
  $\gamma$, \ie $C_\gamma(B_1, \dots, B_n) \bydef{=} \compose[\{C_\gamma,
  B_1, \dots, B_n\}] = \gamma(B_1, \dots, B_n)$, where the latter is the
  composed component obtained by applying the interaction model $\gamma$ to
  behaviours $B_1, \dots, B_n$ with the operational semantics defined in
  \cite{BliSif07-acp-emsoft}.
\end{example}

\begin{example}[Network sort]
  \label{ex:network}
  ...
\end{example}

\begin{definition}[Topologies]
  The {\em topology} of a system $S$ is a directed graph $\topo{S} =
  (S, E)$, having the components of the system as vertices and the set
  of edges $E = \setdef{(T_i,T_j)}{\pout_i \cap \pin_j \neq
    \emptyset}$.  In other words, there is an edge from $T_i$ to $T_j$
  if some of the requests of the former can be granted by the latter.

  If $\topo{S}$ is a directed acyclic graph (DAG), the $S$ is a {\em
  flow-system}.  If $\topo{S}$ is disconnected, \ie $E = \emptyset$, the
  system is {\em independent}.
\end{definition}

Observe that any non-trivial (\ie non-empty) closed flow-system $S$
necessarily contains a certain number of behaviours.  Each conversation can
be assigned a {\em level} defined as the maximal length of a directed path
in $\topo{S}$ from some behaviour to this conversation.

Composition defined by the operational semantics of systems allows
simultaneous execution of interactions authorised by any number of
top-level conversations.  However, mutual exclusion can be ensured by
applying a single top-level arbiter conversation.  Such arbiter
conversation can enforce simple non-deterministic mutual exclusion, as in
\ex{bip}, or any complex scheduling protocol, \eg Round-Robin.

% *********************************************************************

\subsection{Data Management and Transfer in Flow-Systems}
\label{sec:data}

We now extend the definitions of \secn{model} to incorporate data.  To
avoid dealing with causality cycles, we limit ourselves to flow-systems,
where causal data dependencies can be unambigously resolved.  Intuitively,
in each cycle, we propagate the data from behaviours upwards through all
relevant conversations.  At each component, this data can influence the
decision as to what transitions are enabled.  Finally, once a global
interaction has been choosen at the top level, the updated data is
propagated back to behaviours.  Below, we present this in a formal manner.

\begin{definition}[\compmodel{} with data]
  A {\em \compmodel{} with data} is a tuple $(T, \data[i], \data[l],
  \data[o], up, down)$, where $T = (Q, \pin, \pout, \goesto)$ is a
  \compmodel{}, $\data[i]$, $\data[l]$ and $\data[o]$ are respectively the
  domains of {\em input}, {\em local} and {\em output} data, and $up$ and
  $down$ are mappings from the transition relation $\goesto$ to
  functions $(\data[l] \times \data[o])^{\data[i]}$ and $\data$
\end{definition}
% *********************************************************************

\subsection{Expressiveness Results}
\label{sec:expressiveness}


Types of conversations:
\begin{itemize}
\item Linear (finite or cyclic) vs.\ branching.  Branching can be
  \begin{itemize}
  \item deterministic
  \item non-deterministic
  \end{itemize}
\item With or without memory (``unbounded'' data values)
\item One label vs.\ multiple ``send'' labels for interactions composing
  the conversation
\end{itemize}

Conversations (deterministic or not) are strictly more expressive than
interactions.

Deterministic and non-deterministic conversations without memory are
equivalent.

Non-deterministic conversations with memory are strictly more expressive
than deterministic conversations with memory
